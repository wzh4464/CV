\documentclass[11pt,a4paper]{moderncv}
\moderncvstyle{classic}
\moderncvcolor{blue}
\usepackage[utf8]{inputenc}
\usepackage[scale=0.85]{geometry}

%-------------------- Personal Data --------------------
\name{Zihan}{Wu}
\title{PhD Candidate in Electrical Engineering}
\address{City University of Hong Kong}{Hong Kong SAR, China}
\phone[mobile]{(+852) 9810 6427}
\phone[mobile]{(+86) 188 5695 6416}
\email{wzh4464@gmail.com}
\homepage{https://scholar.zihanng.shop}
\social[github]{wzh4464}
\social[linkedin]{zihan-wu-ustc}
\social[orcid]{0000-0002-6551-6177}
\social[researchgate]{Zihan-Wu-13}

\begin{document}
\makecvtitle

%-------------------- Summary --------------------
\section{Summary}
\cvitem{}{
PhD candidate specializing in machine learning, computer vision, and data mining with strong research expertise. Experienced in co-clustering algorithms, privacy-preserving techniques, and unlearning methods. Published researcher with expertise in both theoretical and applied aspects of AI systems, focusing on scalable solutions for large datasets, statistical analysis, and computational optimization applicable to massive data modeling. \textbf{Practical experience training Vision Transformers on multi-GPU clusters and implementing MPI-based co-clustering pipelines.}
}

%-------------------- Technical Skills --------------------
\section{Technical Skills}
\cvitem{Programming Languages}{C++ (advanced), Python (advanced), MATLAB (intermediate)}
\cvitem{Frameworks \& Libraries}{PyTorch, TensorFlow, NumPy, OpenCV, scikit-learn}
\cvitem{Distributed Computing}{MPI, Slurm, PyTorch DDP, CUDA/NCCL}
\cvitem{Tools \& Platforms}{Git, Linux}

%-------------------- Education --------------------
\section{Education}
\cventry{2020--2025 (expected)}{Ph.D.\ in Electrical Engineering}{City University of Hong Kong}{Hong Kong SAR, China}{}{
Thesis: \emph{Adaptive Co-Clustering Algorithm and Applications in Measurement Systems and Biological Imaging}.\\
Advisor: Prof.\ Hong Yan (IEEE Life Fellow).
}
\cventry{2015--2020}{B.Sc.\ in Physics \& Mathematics}{University of Science and Technology of China}{Hefei, Anhui, China}{}{
Double Major: Physics; Mathematics and Applied Mathematics.
}

%-------------------- Selected Projects --------------------
\section{Selected Projects}
\cvitem{\textbf{ViT Unlearning at Scale}}{
Built PyTorch DDP pipelines to unlearn contaminated data shards from a 85M-parameter Vision Transformer, achieving 100-fold reduction in unlearning costs on a 8×V100 node.
}
\cvitem{\textbf{MPI Co-Clustering Framework}}{
Extended Scalabel co-clustering to a 32-node MPI cluster, achieving a 30\% reduction in co-clustering time on 1 TB of text data.
}

%-------------------- Experience --------------------
\section{Experience}
\cventry{Jun.\ 2025}{Incoming AI Intern}{Huawei Hong Kong Research Center}{Hong Kong SAR, China}{}{
AI for software development
}
\cventry{Nov.\ 2024--Present}{Research Assistant}{City University of Hong Kong, Dept.\ of Electrical Engineering}{Hong Kong SAR, China}{}{
    Adaptive Co-Clustering Algorithm and Applications in Measurement Systems and Biological Imaging
}
\cventry{Jun.\ 2018--Sep.\ 2018}{Research Assistant}{University of Oxford, Physics Department}{Oxford, UK}{}{
Investigated single molecular semiconductor structures based on DNA scaffolds and characterized optoelectronic behavior.
}

%-------------------- Publications --------------------
\section{Publications}
\subsection*{Published \& In Press}
\begin{enumerate}\setlength\itemsep{2pt}
  \item Jie Xu, Zihan Wu, Cong Wang, and Xiaohua Jia, “LMEraser: Large model unlearning via adaptive prompt tuning,” to appear in \emph{AISTATS}, 2025.
  \item Zihan Wu, Zhaoke Huang, and Hong Yan, “Scalable co-clustering for large-scale data through dynamic partitioning and hierarchical merging,” in \emph{Proc.\ IEEE SMC}, October 2024.
  \item Zihan Wu, Zhaoke Huang, and Hong Yan, “Ellipse detection via global arc compatibilities and adaptive co-clustering for real-world measurement systems,” \emph{IEEE Trans.\ Instrum.\ Meas.}, 2024.
  \item Zhaoke Huang, Zihan Wu, and Hong Yan, “A convex-hull based method with manifold projections for detecting cell protrusions,” \emph{Computers Biol.\ Med.}, 2024.
  \item Jie Xu, Zihan Wu, Cong Wang, and Xiaohua Jia, ``Machine unlearning: Solutions and challenges,'' \emph{IEEE Transactions on Emerging Topics in Computational Intelligence}, 2024.
  \item Jie Xu, Yulong Ming, Zihan Wu, Cong Wang, and Xiaohua Jia, “X-Shard: Optimistic cross-shard transaction processing for sharding-based blockchains,” \emph{IEEE Trans.\ Parallel Distrib.\ Syst.}, 2024.
\end{enumerate}

\subsection*{Under Review}
\begin{enumerate}\setlength\itemsep{2pt}
  \item Zihan Wu, Zhaoke Huang, and Hong Yan, “DiMergeCo: A Scalable Framework for Large-Scale Co-Clustering with Theoretical Guarantees,” under review.
\end{enumerate}

%-------------------- Awards --------------------
\section{Awards \& Honors}
\cvitem{2020--2024}{Hong Kong PhD Fellowship Scheme (HKPFS).}
\cvitem{2017--2018}{National Encouragement Scholarship, Ministry of Education of China (top 2\%).}

\end{document}