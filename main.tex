\documentclass[caps, english]{financecv}

\name{Zihan Wu}
\phone{(+852) 9810 6427}
\email{wzh4464@gmail.com}
\address{PhD Candidate, Electrical Engineering, City University of Hong Kong}
\homepage{https://scholar.zihanng.shop}
\linkedin{zihan-wu-ustc}

\begin{document}

\begin{cvsection}{Summary}
    Ph.D. candidate in Machine Learning Systems and High-Performance Computing, with expertise in large-scale distributed AI infrastructure and real-time model optimization. Experienced in accelerating Transformer and LLM workloads with C++, Python, and CUDA for low-latency inference and training. Applied advanced techniques such as prompt engineering, RAG, and co-clustering for applications in computer vision and code intelligence.
\end{cvsection}

\begin{cvsection}{Education}
    \cvblock{2020 -- 2025 (Expected)}
    {Hong Kong SAR, China}
    {City University of Hong Kong}
    {}
    {Ph.D. in Electrical Engineering}
    {}
    {}
    {}
    \cvblock{2015 -- 2020}
    {Hefei, China}
    {University of Science and Technology of China}
    {}
    {B.Sc. in Physics and Mathematics and Applied Mathematics}
    {School of the Gifted Young}
    {}
    {}
\end{cvsection}

\begin{cvsection}{Experience}
    \cvblock{2025 -- Present}
    {Hong Kong}
    {Huawei Technologies}
    {AI Research Intern}
    {Optimized LLM prompts for automated code review using RAG and context engineering. Enabled accurate, low-latency analysis by dynamically injecting relevant codebase context.}
    {}
    {}
    {}
    \cvblock{2024 -- Present}
    {Hong Kong}
    {City University of Hong Kong}
    {Research Assistant}
    {Lead engineer on high-performance ML algorithms and latency-sensitive AI systems. Focused on AI system optimization for real-time applications.}
    {}
    {}
    {}
\end{cvsection}

\begin{cvsection}{Projects}
    \project{2023 -- Present}
    {LMEraser: An Exact Unlearning System for Large Models}
    {Python, PyTorch, CUDA}
    {}
    {
        \cvbullet{Designed an exact machine unlearning system for large vision models (e.g., 86M ViT) based on adaptive prompt tuning. }
        \cvbullet{Reduced unlearning computational cost by over 100x compared to retraining by isolating private data influence into tunable prompts and classifier heads, while keeping the main model frozen. }
        \cvbullet{Engineered a private data clustering mechanism to generate tailored prompts for distinct data groups, preserving high model utility after unlearning. }
        \cvbullet{Presented at AISTATS 2025. }
    }
    {}
    \project{2020 -- Present}
    {High-Performance ML Clustering System}
    {C++, MPI, Rust, Python, OpenCV}
    {}
    {
        \cvbullet{Designed a scalable co-clustering framework for large-scale matrix data and real-world vision applications.}
        \cvbullet{Accelerated document and recommendation system co-clustering using a distributed MPI backend.}
        \cvbullet{Applied adaptive co-clustering for ellipse detection in noisy and occluded images, improving robustness and accuracy.}
        \cvbullet{Outperformed state-of-the-art methods with 10\% higher detection accuracy and 15\% recall gain on real-world datasets.}
        \cvbullet{Published in IEEE SMC 2024 and IEEE TIM 2025. }
    }
    {}
    \project{2023 -- 2024}
    {X-Shard: Distributed AI Transaction Engine}
    {C++, Distributed Systems, Performance Engineering}
    {}
    {
        \cvbullet{Reduced transaction latency by 37\% using optimized commit paths for sharded workloads}
        \cvbullet{Designed cache-aligned, branch-minimized data layout for consistent real-time inference}
        \cvbullet{Published: IEEE Trans. on Parallel and Distributed Systems, 2024}
    }
    {}


\end{cvsection}

\begin{cvsection}{Skills}
    \skillgroup{Languages}{Python, C++ (Advanced), Rust, CUDA}
    \skillgroup{AI \& ML}{Deep Learning, Transformers, LLM Fine-tuning, Prompt Engineering, Computer Vision, Representation Learning}
    \skillgroup{Systems \& Infrastructure}{Distributed Systems, AI Infrastructure, HPC, MPI, OpenMP, Blockchain}
    \skillgroup{Frameworks \& Tools}{PyTorch (Advanced), OpenCV, Linux, Docker, Git}
\end{cvsection}

\begin{cvsection}{Awards}
    \award{Hong Kong PhD Fellowship}{Top 5\% acceptance rate}{2020--2024}
    \award{National Encouragement Scholarship}{Top 3\% nationwide}{2017--2018}
    \award{Patent: Physical Activity Assessment System}{Patent ID: HK30081186}{2023}
\end{cvsection}

\end{document}